\documentclass[12pt,a4paper]{exam}
    \usepackage{amsmath,amsthm,amsfonts,amssymb}
    % \usepackage{tabular}
   
    \newcounter{matchleft}
    \newcounter{matchright}

    \newenvironment{matchtabular}{%
        \setcounter{matchleft}{0}%
        \setcounter{matchright}{0}%
        \tabularx{\textwidth}{%
            >{\leavevmode\hbox to 1.5em{\stepcounter{matchleft}\arabic{matchleft}.}}X%
            >{\leavevmode\hbox to 1.5em{\stepcounter{matchright}\alph{matchright})}}X%
            }%
    }

    \setlength\parindent{0pt}
        %usage \choice{ }{ }{ }{ }
        %(A)(B)(C)(D)
        \newcommand{\fourch}[4]{
        \par
                \begin{tabular}{*{4}{@{}p{0.23\textwidth}}}
                (1)~#1 & (2)~#2 & (3)~#3 & (4)~#4
                \end{tabular}
        }

        %(A)(B)
        %(C)(D)
        \newcommand{\twoch}[4]{

                \begin{tabular}{*{2}{@{}p{0.46\textwidth}}}
                (1)~#1 & (2)~#2
                \end{tabular}
        \par
                \begin{tabular}{*{2}{@{}p{0.46\textwidth}}}
                (3)~#3 & (4)~#4
                \end{tabular}
        }

        %(A)
        %(B)
        %(C)
        %(D)
        \newcommand{\onech}[4]{
          \begin{tabular}{*{1}{@{}p{0.46\textwidth}}}
          (1)~#1 &
          \end{tabular}
          \begin{tabular}{*{1}{@{}p{0.46\textwidth}}}
          (2)~#2 &
          \end{tabular}
          \begin{tabular}{*{1}{@{}p{0.46\textwidth}}}
          (3)~#3 &
          \end{tabular}
          \begin{tabular}{*{1}{@{}p{0.46\textwidth}}}
          (4)~#4 &
          \end{tabular}
        }

        \newlength\widthcha
        \newlength\widthchb
        \newlength\widthchc
        \newlength\widthchd
        \newlength\widthch
        \newlength\tabmaxwidth

        \setlength\tabmaxwidth{0.96\textwidth}
        \newlength\fourthtabwidth
        \setlength\fourthtabwidth{0.25\textwidth}
        \newlength\halftabwidth
        \setlength\halftabwidth{0.5\textwidth}

      \newcommand{\choice}[4]{%
      \settowidth\widthcha{AM.#1}\setlength{\widthch}{\widthcha}%
      \settowidth\widthchb{BM.#2}%
      \ifdim\widthch<\widthchb\relax\setlength{\widthch}{\widthchb}\fi%
      \settowidth\widthchb{CM.#3}%
      \ifdim\widthch<\widthchb\relax\setlength{\widthch}{\widthchb}\fi%
      \settowidth\widthchb{DM.#4}%
      \ifdim\widthch<\widthchb\relax\setlength{\widthch}{\widthchb}\fi%
      \ifdim\widthch<\fourthtabwidth
        \fourch{#1}{#2}{#3}{#4}
      \else\ifdim\widthch<\halftabwidth
        \ifdim\widthch>\fourthtabwidth
          \twoch{#1}{#2}{#3}{#4}
        \else
          \onech{#1}{#2}{#3}{#4}
        \fi
      \fi\fi
    }
                
    
    \title{\vspace{-4em}Random Questions}
    \date{\vspace{-3em}}
    \everymath{\displaystyle}
    \begin{document}
    \maketitle
      \begin{questions}
        \question{Let $f(x)$ be defined as: $$f(x)=\begin{cases}
            x^2\sin\left(\frac{1}{x}\right) &  x \not= 0 \\
            0 & x = 0
          \end{cases}$$ \\
          If the function $ f(x) $ is continuos then $ f^\prime(0) $ is:}
          \choice{0}{1}{$-1$}{Not defined}


        \question{The number of integral roots of the equation $$x^8 -24x^7 -18x^5 + 39x^2 + 1155 = 0$$}
        \choice{0}{2}{4}{6}

        \question{Let $x_1, x_2, x_3, \dots, x_k $ be the divisors of positive number $n$ (including 1 and $n$). If $x_1 + x_2 + x_3+ \dots + x_k = 2022$, then $\sum^k_{i=1} \left(\frac{1}{x_i} \right)$ is equal to: }
        \choice{$\frac{2022}{k}$}{$\frac{2022}{n}$}{$\frac{1}{n}$}{$\frac{1}{2022}$}

        
        \question{Suppose that $f$ satisfies the equation $f(x+y)=f(x)+f(y)+x^2y+xy^2$ $\forall x,y\in R$. Suppose further that $$ \lim_{x \to 0} \frac{f(x)}{x} = 1 $$ Find $f^\prime(x)$}

        \question{Solve for $x$, $$ x^{\ln x} = x\ln x $$}



        \question{Consider the function $f(x) = a^{a^x} - x$. If the equation $f(x)=0$ has exactly 2 roots. Then range of $a$ is}
        \choice{$a \in \left(0,1\right)$}{$a \in \left(1, e^{\frac{1}{e}}\right)$}{ $a \in \left(0,1\right) \cup  \left(1, e^{\frac{1}{e}}\right)$}{No such $a$ exists}


        \question{Solve for $x$, $$ 2^{x^6} + 2^{x^2} = 2^{x^4 + 1} $$}

        

        

      \end{questions}
\end{document}


   
