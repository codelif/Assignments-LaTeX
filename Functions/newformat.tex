\documentclass[12pt,a4paper]{exam}
    \usepackage{amsmath,amsthm,amsfonts,amssymb,dsfont}
    % \usepackage{tabular}
   
    \newcounter{matchleft}
    \newcounter{matchright}

    \newenvironment{matchtabular}{%
        \setcounter{matchleft}{0}%
        \setcounter{matchright}{0}%
        \tabularx{\textwidth}{%
            >{\leavevmode\hbox to 1.5em{\stepcounter{matchleft}\arabic{matchleft}.}}X%
            >{\leavevmode\hbox to 1.5em{\stepcounter{matchright}\alph{matchright})}}X%
            }%
    }

    \setlength\parindent{0pt}
        %usage \choice{ }{ }{ }{ }
        %(A)(B)(C)(D)
        \newcommand{\fourch}[4]{
        \par
                \begin{tabular}{*{4}{@{}p{0.23\textwidth}}}
                (1)~#1 & (2)~#2 & (3)~#3 & (4)~#4
                \end{tabular}
        }

        %(A)(B)
        %(C)(D)
        \newcommand{\twoch}[4]{

                \begin{tabular}{*{2}{@{}p{0.46\textwidth}}}
                (1)~#1 & (2)~#2
                \end{tabular}
        \par
                \begin{tabular}{*{2}{@{}p{0.46\textwidth}}}
                (3)~#3 & (4)~#4
                \end{tabular}
        }

        %(A)
        %(B)
        %(C)
        %(D)
        \newcommand{\onech}[4]{
        \par
              (1)~#1 \par (2)~#2 \par (3)~#3 \par (4)~#4
        }

        \newlength\widthcha
        \newlength\widthchb
        \newlength\widthchc
        \newlength\widthchd
        \newlength\widthch
        \newlength\tabmaxwidth

        \setlength\tabmaxwidth{0.96\textwidth}
        \newlength\fourthtabwidth
        \setlength\fourthtabwidth{0.25\textwidth}
        \newlength\halftabwidth
        \setlength\halftabwidth{0.5\textwidth}

      \newcommand{\choice}[4]{%
      \settowidth\widthcha{AM.#1}\setlength{\widthch}{\widthcha}%
      \settowidth\widthchb{BM.#2}%
      \ifdim\widthch<\widthchb\relax\setlength{\widthch}{\widthchb}\fi%
      \settowidth\widthchb{CM.#3}%
      \ifdim\widthch<\widthchb\relax\setlength{\widthch}{\widthchb}\fi%
      \settowidth\widthchb{DM.#4}%
      \ifdim\widthch<\widthchb\relax\setlength{\widthch}{\widthchb}\fi%
      \ifdim\widthch<\fourthtabwidth
        \fourch{#1}{#2}{#3}{#4}
      \else\ifdim\widthch<\halftabwidth
        \ifdim\widthch>\fourthtabwidth
          \twoch{#1}{#2}{#3}{#4}
        \else
          \onech{#1}{#2}{#3}{#4}
        \fi
      \fi\fi
    }
                
    
    \title{\vspace{-4em}Functions Assignment}
    \date{\vspace{-3em}}
    \everymath{\displaystyle}
    \begin{document}
    \maketitle
        \begin{questions}
        \question{If \(f(x)=\pi\left(\frac{\sqrt{x+7}-4}{x-9}\right)\), then range of function \(y = \sin{(2f(x))}\) is}
        \choice {$\left[0, 1\right]$}{$\left(0,\frac{1}{\sqrt{2}}\right]$}{$\left(0,\frac{1}{\sqrt{2}}\right) \cup \left(\frac{1}{\sqrt{2}}, 1\right]$}{$\left(0, 1\right]$}
  
        \question{If the range of function \(f(x)=\frac{x^2+x+c}{x^2+2x+c}\), \(x \in R\) is \(\left[\frac{5}{6}, \frac{3}{2}\right]\), then \(c\) is equal to}
        \choice{$-4$}{$3$}{$4$}{$5$}
        
        \question{If a polynomial function $f$ satisfies the relation $$\log_2\left[f(x)\right]=\log_2\left(2+\frac{2}{3}+\frac{2}{9}+...+\infty\right)\cdot \log_3\left(1+\frac{f(x)}{f\left(\frac{1}{x}\right)}\right)$$ and $f(10) = 1001$, then value of $f(20)$ is}
        \choice{2002}{7999}{8001}{16001}

        \question{Consider, $P=\frac{x^2-2x}{x^2+x+1}$, $Q=\frac{y-1}{y^2+y+1}$, $R=\frac{2}{z^2+z+1}$ where $x,y,z \in R$.\vspace{4.5mm} If $k=\left[P+Q+R\right]-\left(\left[P\right]+\left[Q\right]+\left[R\right]\right)$ then the possible value(s) of $k$ is(are) (where $\left[\cdot\right]$ denotes greatest integer less than equal to x)}
        \choice{0}{1}{2}{3}
        
        \question{Let $f$ be a function defined in $\left[-2, 3\right]$ given as $$ f(x)=
        \begin{cases}
            3(x+1)^{1/3}, & -2\leq x < 0 \\
            -(x-1)^2, & 0\leq x < 1 \\
            2(x-1)^2 & 1\leq x < 2 \\
            -x^2+4x-3, &2\leq x \leq 3
        \end{cases}$$ }

        
        \begin{center}
            \begin{tabular}{l p{0.54\linewidth} l p{0.35\linewidth}}

                & \textbf{Column I} && \textbf{Column II}\\
                &&&\\
                (A)  &   The number of integers in the range of $f(x)$ is &   (p)   &   2\\
                &&&\\
                (B)  &   The number of integral values of x which are in the domain of $f(1-\left|x\right|)$, is   &   (q)   &   4\\
                &&&\\
                (C)  &   The number of integers in the range of $\left|f(-\left|x\right|)\right|$, is      &   (r)   &   6\\
                &&&\\
                (D)  &   The number of integral values of k for which the equation $f(\left|x\right|) = k$ has exactly four distinct solutions is      &   (s)   &   7
            \end{tabular}
            \end{center}
        \question{Let $f(x) = \left|x^2-9\right|-\left|x-a\right|$. Find the number of integers in the range of $a$ so that $f(x) = 0$ has 4 distinct real roots.}
    
        \question{The set of real values of $x$ satisfying the equality $\left[\frac{3}{x}\right]+\left[\frac{4}{x}\right] = 5$ (where $\left[\cdot\right]$ denotes the greatest integer function) belongs to the inteval $\left(a,\frac{b}{c}\right]$ where $a,b,c \in N$ and $\frac{b}{c}$ is in its lowest form. Find the value of $a+b+c+abc$.}
        
        \question{Let $R$ be the set of real numbers and $f:R\to R$, be a differentiable function such that $|f(x)-f(y)| \leq |x-y|^3  \hspace{5pt}\forall\hspace{5pt} x,y \in R$. If $f(10) = 100$, the value of $f(20)$ is equal to}
        \choice{0}{20}{100}{10}

        \question{If the equation $$\left||x-1|-6 \lim_{t \to \infty} \left(\frac{\sqrt{2t^2-t-1}-\sqrt{t^2-t+1}}{t\left(\tan{\frac{\pi}{8}}\right)}\right)\right| = k$$ has four distinct solutions then find the number of integral values of $k$.}
        
        \question{Let $A$ and $B$ be two sets containing 2 and 3 elements respectively. Then, total number of subsets of $A\times B$ having 3 or more elements is}
        \choice{42}{56}{54}{52}

        \question{Let the sets be $A = \{x:x \in Z^+$ and $x \leq 9\}, B=\{x:x \in Z$ and $-3 < x < 8\}$ and $C = \{x:x$ is a prime number$\}$, then the number of elements belonging to exactly two of the three sets $A, B$ and $C$ is }
        \choice{3}{4}{6}{8}

        \question{Let $f$ be a function from non-negative integers to non-negative integers such that $f(xy) = xf(y) + yf(x)$. It is given that $f(10)=19$, $f(12)=52$ and $f(15)=26$, then $f(8)$ is equal to}
        \choice{26}{36}{38}{40}

        \question{If a function $f(x)$ satisfies the relation $3f(x)-5f\left(\frac{2}{x}\right)=3-x+x^2$ $\forall$ $x \in R-\{0\}$, then the value of $f(1)$ is equal to}
        \choice{$-2$}{$\frac{-33}{16}$}{$\frac{-17}{8}$}{$\frac{-35}{16}$}

        \question{The function $f(x) = \sqrt{\log_2\log_3\log_4 x} + \sqrt{\{x^2+x+1\}}$ (where \{\} represents fractional function, is well defined). Then $x$ may belong to}
        \choice{$\left[0, 32\right]$}{$\left[64, \infty\right)$}{$\left[100, \infty\right)$}{$\left[0, 64\right)$}

        \question{Let $g$ be a function satisfying $g(x-2)+g(x+2)=\sqrt{3}g(x),$ $\forall$ $x \in R$. Then $g$ also satisfies the relation(s) given as}
        \choice{$g(x-4)+g(x+8)=0$}{$g(x)-g(x+12)=0$}{$g(x-2)+g(x+4)=0$}{$g(x)-g(x+24)=0$}
        
        \question{Domain of the function $f(x)=\log_{\left[x-1\right]}(\sin^2\pi x)$ is $x \in \left[3, \infty\right) -\{p$, $p+1$, $p+2$, $\dots$, $n\}$, where $\left[\cdot\right]$ denotes greatest integer function and $n$ is a multiple of 4 and 25, then $(p)+(p+1)+(p+2)+\dots + n$ can be}
        \choice{5047}{20097}{5050}{45147}

        \question{Let $f$ be a real valued function which satisfies $f\left(\frac{x-3}{x+1}\right)+f\left(\frac{3+x}{1-x}\right) = x$, $|x| \not= 1,$ $x\in R$. Then $f(3)$ is equal to}
        \choice{$-3$}{3}{$-\frac{11}{3}$}{$\frac{11}{3}$} 
        
        \question{Let $f(x^2-11x+10)+f(x^2-21x+20)=x^4-3x^2 +21x+2$, then the value of $f(0)$ is}
        \choice{20}{$\frac{21}{2}$}{24}{0}

        \question{The range of the function $f(x) = \frac{\sin(\pi\left[x^2+x+1\right])}{\cos(\pi\left[x^2+3x+2\right])}$ is ($\left[\cdot\right]$ represents greatest integer function)}
        \choice{$\{0\}$}{$\{1,-1\}$}{$\left[-1,1\right]$}{$\left(-\infty,\infty\right)$}

        \question{If a function satisfies $$(x-y)f(x+y)-(x+y)f(x-y)=2(x^2y-y^3) \hspace{5pt}\forall\hspace{5pt}x,y \in R \hspace{5pt}and\hspace{5pt}f(1)=5$$ then}
        \choice{$f(x)$ is not differentiable}{$f(x) = x^2+4x$}{$f(0)=1$}{$f'(1)=8$}

        \question{Total number of solutions of $\sin\{x\}=\cos\{x\}$ (where $\{\cdot\}$ denotes the fractional part) in $\left[0,2\pi\right]$ is}

        \question{Set of exhaustive values of $x$ satisfying $$\left||x|^2-x+1\right|>\left|x^2-1\right|$$}
        \choice{$\left(-\infty,0\right)\cup\left(\frac{1}{2}, \infty\right)$}{$\left(0,2\right)$}{$\left(\frac{1}{2},2\right)$}{$\left(-\infty,0\right)\cup\left(\frac{1}{2}, 2\right)$}

        \question{Consider an equation $Sgn(\left[x\right])+\lambda = x$. (where $[\cdot]$ represents greatest integer function). if $n_\lambda$ denotes the number of solutions of the equation, then the value of $\frac{n_{\frac{1}{2}}}{n_1}$ is}
        \choice{1}{$\frac{3}{2}$}{3}{$\frac{1}{2}$}

        \question{The solution of the equation $2[x]+[3x]=6$ (where $[\cdot]$ denotes the greatest integer function) is}
        \choice{$\left[\frac{4}{3}, \frac{5}{3}\right)$}{$\left[\frac{5}{3}, 2\right)$}{$\left[\frac{2}{3}, \frac{5}{3}\right)$}{$\left[1, \frac{5}{3}\right)$}

        \question{$A=\{x \in N :$ HCF$(x,12)=1$, $x < 12\}$, $B= \{x \in N:$ LCM$(x,12)=12\}$, then the number of relations from $A$ to $C$, where $A\Delta C=B$, is}
        \choice{$2^{28}$}{$2^{32}$}{$2^{24}$}{$2^{36}$}

        \question{The number of subsets of a set with 2018 elements having an odd number of elements is}
        \choice{$2^{2015}$}{$2^{2016}$}{$2^{2017}$}{Data insufficient}

        \question{The range of the function $f(x)=\frac{(x+3)^2}{x^2+1}$ is}
        \choice{$\left[0,12\right]$}{$\left[0,11\right]$}{$\left[0,10\right]$}{$\left[0,15\right]$}
        \question{The maximum value of $f(x)= | 15-8x+|x|^2|$ in the inteval $(3.5,4.5)$ is}
        \choice{0}{1}{2}{3}
        
        \question{If $X$ and $Y$ are two sets such that $$n(X \cap \overline{Y}) = 12,\hspace{5pt} n(\overline{X} \cap Y) =15\hspace{5pt} and\hspace{5pt} n(X \cup Y) = 30$$ then $n(X \times Y) = $}
        \choice{210}{270}{180}{300}

        \question{Let $3^{f_1(x)}+3^x = 9$ and $f_2(x)= \log_{\frac{1}{2}}(a+2x-x^2)$. If maximum integral value of $f_1(x)$ is equal to the minimum value of $f_2(x)$, then $a$ is equal to}
        \choice{$-1$}{$-\frac{1}{2}$}{Zero}{1}

        \question{Find the range of the function $$f(x) = \log_{\frac{1}{2}}(2\sin^2x - 2\sin x +1)$$}
        \choice{$\left[\log_{\frac{1}{2}}5,0\right]$}{$\left[\log_{\frac{1}{2}}5,1\right]$}{$\left(-\infty,\log_{\frac{1}{2}}5\right]$}{$\left[\log_{\frac{1}{2}}5,\infty\right)$}
    \end{questions}
\end{document}

    % \end{document}

   
