\documentclass[12pt,a4paper]{exam}
    \usepackage{amsmath,amsthm,amsfonts,amssymb,dsfont}
    % \usepackage{tabular}
   
    \newcounter{matchleft}
    \newcounter{matchright}

    \newenvironment{matchtabular}{%
        \setcounter{matchleft}{0}%
        \setcounter{matchright}{0}%
        \tabularx{\textwidth}{%
            >{\leavevmode\hbox to 1.5em{\stepcounter{matchleft}\arabic{matchleft}.}}X%
            >{\leavevmode\hbox to 1.5em{\stepcounter{matchright}\alph{matchright})}}X%
            }%
    }

    \setlength\parindent{0pt}
        %usage \choice{ }{ }{ }{ }
        %(A)(B)(C)(D)
        \newcommand{\fourch}[4]{
        \par
                \begin{tabular}{*{4}{@{}p{0.23\textwidth}}}
                (1)~#1 & (2)~#2 & (3)~#3 & (4)~#4
                \end{tabular}
        }

        %(A)(B)
        %(C)(D)
        \newcommand{\twoch}[4]{

                \begin{tabular}{*{2}{@{}p{0.46\textwidth}}}
                (1)~#1 & (2)~#2
                \end{tabular}
        \par
                \begin{tabular}{*{2}{@{}p{0.46\textwidth}}}
                (3)~#3 & (4)~#4
                \end{tabular}
        }

        %(A)
        %(B)
        %(C)
        %(D)
        \newcommand{\onech}[4]{
          \begin{tabular}{*{1}{@{}p{0.46\textwidth}}}
          (1)~#1 &
          \end{tabular}
          \begin{tabular}{*{1}{@{}p{0.46\textwidth}}}
          (2)~#2 &
          \end{tabular}
          \begin{tabular}{*{1}{@{}p{0.46\textwidth}}}
          (3)~#3 &
          \end{tabular}
          \begin{tabular}{*{1}{@{}p{0.46\textwidth}}}
          (4)~#4 &
          \end{tabular}
        }

        \newlength\widthcha
        \newlength\widthchb
        \newlength\widthchc
        \newlength\widthchd
        \newlength\widthch
        \newlength\tabmaxwidth

        \setlength\tabmaxwidth{0.96\textwidth}
        \newlength\fourthtabwidth
        \setlength\fourthtabwidth{0.25\textwidth}
        \newlength\halftabwidth
        \setlength\halftabwidth{0.5\textwidth}

      \newcommand{\choice}[4]{%
      \settowidth\widthcha{AM.#1}\setlength{\widthch}{\widthcha}%
      \settowidth\widthchb{BM.#2}%
      \ifdim\widthch<\widthchb\relax\setlength{\widthch}{\widthchb}\fi%
      \settowidth\widthchb{CM.#3}%
      \ifdim\widthch<\widthchb\relax\setlength{\widthch}{\widthchb}\fi%
      \settowidth\widthchb{DM.#4}%
      \ifdim\widthch<\widthchb\relax\setlength{\widthch}{\widthchb}\fi%
      \ifdim\widthch<\fourthtabwidth
        \fourch{#1}{#2}{#3}{#4}
      \else\ifdim\widthch<\halftabwidth
        \ifdim\widthch>\fourthtabwidth
          \twoch{#1}{#2}{#3}{#4}
        \else
          \onech{#1}{#2}{#3}{#4}
        \fi
      \fi\fi
    }
                
    
    \title{\vspace{-4em}Complex \& Quadratic Assignment}
    \date{\vspace{-3em}}
    \everymath{\displaystyle}
    \begin{document}
    \maketitle
      \begin{questions}

        \question{If the quadratic expression $px^2+ |2p-3|x-6$ is positive for exactly two integral values of $x$ then (where $[\cdot]$ is greatest integer function)}
        \choice{$[p]=-1$}{$[p]=-2$}{$p\in\left(-\frac{3}{4},-\frac{3}{5}\right]$}{$p\in(-2,-1]$}
      
        \question{If $lx^{17}+mx^{16}+1$ is divisible by $x^2-x-1$ then}
        \choice{$l$ is divisible by 3}{$l$ is divisible by 7}{$l$ is divisible by 47}{$l$ is divisible by 21}
        
        \vspace{3mm}
        \hrule
        \vspace{3mm}
        \item[\textbf{Question:}]{Let $\alpha$, $\beta$, $\gamma$, $\delta$ be the roots of $x^4+ax^3+bx^2+cx+d=0$ if $(\alpha + \beta) = (\gamma + \delta)$ and $a$, $b$, $c$, $d$ $\in R$, then}

        \question{The correct options is/are}
        \choice{If $a=2$, then $b-c \not=2$}{If $a=2$, then $b-c =1$}{If $a=1$, then $b-2c \not=1$}{If $a=1$, then $b-2c =\frac{1}{4}$}
        \question{If $b+c=1$ and $a \not= -2$ then}
        \choice{$b \leq \frac{3}{4}$}{$b \geq \frac{3}{4}$}{$c \leq \frac{1}{4}$}{$c \geq \frac{1}{4}$}
        
        \vspace{3mm}
        \hrule
        \vspace{3mm}
        \item[\textbf{Question:}]{For any complex number $z$, $z = |z|\left[\cos(\arg(z))+\dot{\imath}\sin(\arg(z))\right]$. Choose the correct answer(s)}
        \question{If $z$ is any non-zero complex number, then}
        \choice{$\left|\frac{z}{|z|}-1\right| \leq \left|\arg(z)\right|$}{$\left|\frac{z}{|z|}-1\right| > \left|\arg(z)\right|$}{$|z-1|>||z|-1|+|z||\arg(z)|$}{$|z-1|\leq||z|-1|+|z||\arg(z)|$}

        \question{If $z$ is any non-zero complex number such that $\arg\left(z^{\frac{3}{8}}\right) = \frac{1}{2}\arg\left(z^2+\overline{z}z^\frac{1}{2}\right)$, then $z$}
        \begin{flushleft}
          {(1) Must be purely imaginary and non-unimodular complex number}\vspace{1mm}\break
          \vspace{1mm}{(2) Could be unimodular complex number}\break
          \vspace{1mm}{(3) Could be purely real complex number}\break
          \vspace{1mm}{(4) Must be purely imaginary complex number}
    
        \end{flushleft}
        \hrule
        \vspace{3mm}
        \item[\textbf{Question:}] {Let $ e^{\ln[1+ \left\{xyz\right\}]}$, $\log_yx$, $\log_zy$ and $\log_xz^{-15}$ be the first four terms of an A.P. with common difference $d$, where all terms of the A.P. are real and defined (where $\left[\cdot\right]$ and $\left\{\cdot\right\}$ represents greatest integer function and fractional part function) then answer the following questions}
        
        \question{Which of the following interval(s) contain $d$?}
        \choice{$[0, \infty)$}{$(-\infty,0]$}{$[-10,20]$}{$[10,20]$}

        \question{The value of $\sum_{k=1}^\infty \frac{k}{\left(\frac{x}{z^3}+xy+yz^3\right)^k}$ is less than or equal to}
        \choice{1}{2}{$\frac{1}{2}$}{$\frac{2}{3}$}
        
        \vspace{3mm}
        % \begin{flushleft}
        
        \hrule
          \item[\textbf{Question:}]{$\alpha$, $\beta$, $\gamma$ are the roots of a cubic equation $ax^3+bx^2+cx+d=0$ then $\alpha+\beta+\gamma=-\frac{b}{a}$, $\alpha\beta+\beta\gamma+\gamma\alpha=\frac{c}{a}$, $\alpha\beta\gamma=-\frac{d}{a}$.}\\
        {Let the number $p$ and $q$ are positive and the roots of the equation $x^3-4px+3q=0$ are real. Let $\alpha$ is a root of this given cubic equation of minimum absolute value then}
        % \end{flushleft}
        

        \question{The roots of the given cubic equation must be}
        \choice{Two positive and one negative}{All positive}{All negative}{Two negative and one positive}

        \question{The range of $\alpha$ is}
        \choice{$-\frac{3q}{4p}<\alpha<\frac{q}{2p}$\hspace{5mm}} % Redundant space to force 2x2 answer grid
        {$-\frac{9q}{8p}<\alpha<\frac{3q}{4p}$}
        {$\frac{3q}{4p}<\alpha<\frac{9q}{8p}$}
        {$\frac{q}{p}<\alpha<\frac{9q}{8p}$}
        \vspace{3mm}
        \hrule
        \vspace{3mm}
        \item[\textbf{Question:}]{If $x_1$, $x_2$, $x_3 \dots x_n$ are all positive and $m\in R$, then $$\frac{x_1^m+x_2^m+x_3^m+\dots+x_n^m}{n} \geq \left(\frac{x_1+x_2+x_3+\dots+x_n}{n}\right)^m \text{if } m\in R-(0,1)$$ \begin{center}
          and
        \end{center} $$\frac{x_1^m+x_2^m+x_3^m+\dots+x_n^m}{n} \leq \left(\frac{x_1+x_2+x_3+\dots+x_n}{n}\right)^m \text{if } 0<m<1$$ also, $\frac{x_1+x_2+x_3+\dots +x_n}{n} \geq \left( x_1\cdot x_2\cdot x_3\cdots x_n \right)^\frac{1}{n} \geq \frac{n}{\frac{1}{x_1}+\frac{1}{x_2}+\frac{1}{x_3}+\dots+\frac{1}{x_n}}$. Then answer the following questions.}
        % \vspace{15mm}
        \question{If $x_1>0$, $x_2>0$, $x_3>0$ and $x_1 +x_2 +x_3 = 1$, then the minimum value of $$\frac{x_1}{3-x_1}+\frac{x_2}{3-x_2}+\frac{x_3}{3-x_3}\text{ is}$$}
        \choice{$\frac{3}{8}$}{$\frac{5}{8}$}{$\frac{7}{8}$}{$\frac{3}{4}$}

        \question{If $x_1 + x_2 + x_3 = 3$ and $x_1>0$, $x_2>0$, $x_3>0$ then the minimum value of $$\left(\frac{3}{x_1}-1\right)\left(\frac{3}{x_2}-1\right)\left(\frac{3}{x_3}-1\right) \text{ is}$$}
        \choice{3}{4}{7}{8}
        \vspace{3mm}
        \hrule
        \vspace{3mm}
        \item[\textbf{Question:}]{Let the cubic equation $ax^3+bx^2+cx+d=0$ is shown by $P(a$, $b$, $c$, $d$, $x)=0$. Now answer the following questions.}
        \question{If the equation $P(1$, $0$, $1$, $7$, $x)=0$ has roots $x_1$, $x_2$, $x_3$, then the equation whose roots are $(x_1-x_2)^2$, $(x_2-x_3)^2$, $(x_3-x_1)^2$ is}
        \choice{$P($1, 6, 9, 1327, $x)=0$}
        {$P($1, 1, 9, 127, $x)=0$}
        {$P($1, 6, 9, 127, $x)=0$}
        {$P($6, 9, 1, 127, $x)=0$}

        \question{If the roots of the equation $P(6$, $-11$, $6$, $-1$, $x)=0$ are in HP and equation \\ $P(1$, $-6$, $11$, $-6$, $x)=0$ have roots $\alpha$, $\beta$, $\gamma$, then $\alpha+\beta^2+\gamma^3$ can be}
        \choice{27}{9}{32}{12}

        \vspace{3mm}
        \hrule
        \vspace{3mm}
        \item[\textbf{Question:}]{If Locus of point $P(z)$ in complex plane is $|z+z_1| + |z+z_2|=4$ where $A$ represents $z_1$, as (1, 0) and $B$ represents $z_2$ as $(-1$, $0)$ and $Q(w)$ is variable point inside the locus of $P$ such that all internal angle bisectors of triange $PAB$ concurrent at $Q$ and if $|w-w_1|+|w-w_2|=2$, then}
        
        \question{The value of \(|w_1|+|w_2|\) is equal to}
        \choice{$\frac{2}{\sqrt{3}}$}
        {$\sqrt{\frac{2}{3}}$}
        {$2\sqrt{\frac{2}{3}}$}
        {$\frac{2\sqrt{2}}{3}$}
        \question{If minimum value of $|w-z_1|+|w-z_2|$ is equal to m, then $[m]$ (where $[x]$ denotes greatest integer function) is}
        \choice{1}{2}{3}{4}

        \vspace{3mm}
        \hrule
        \vspace{3mm}
        \item[\textbf{Question:}]{Let $f(x) = x^3+x+1$ suppose $g$ is a cubic polynomial such that $g(0)=-1$ and the roots of $g(x)=0$ are square of the roots of $f(x)=0$. Then}
        \question{The equation $f(x)=0$ has}
        \choice{At least one positive root}
        {At least two negative roots}
        {Exactly one negative root}
        {Exactly two positive roots}

        \question{The polynomial $g(x^2)$ is identical with}
        \choice{$f(x^2)$}
        {$(f(x))^2$}
        {$2f(x)f(-x)$}
        {$-f(x)f(-x)$}

        \vspace{3mm}
        \hrule
        \vspace{3mm}
        \question{If the equation $|z|(z+1)^8=z^8|z+1|$ where $z\in C$ and $z(z+1) \not=0$ has distinct roots $z_1$, $z_2$, $\dots z_n$ (where $n\in N$) then which of the following is/are true?}
        \choice{$z_1$, $z_2$, $\dots z_n$ are concyclic point}
        {$z_1$, $z_2$, $\dots z_n$ are collinear point}
        {$\sum_{r=1}^n\text{Re}(z_r)=-\frac{7}{2}$}
        {$\sum_{r=1}^n\text{Im}(z_r)=0$}

        \question{Let $z$ be a complex number satisfying equation $z^n=(\overline{z})^m$ where $n\text{, }m \in N$ then} \vspace{2mm}\\
        \vspace{1mm}{(1) If $n=m$ then number of solutions will be finite} \\
        \vspace{1mm}{(2) If $n=m$ then number of solutions will be infinite} \\
        \vspace{1mm}{(3) If $n\not=m$ then number of solutions will be $n+m$} \\
        \vspace{1mm}{(4) If $n\not=m$ then number of solutions will be $n+m+1$}

        \vspace{3mm}
        \hrule
        \vspace{3mm}
        \item[\textbf{Question:}]{Consider the quadratic equation $ax^2 - (2a-1)x+a-3=0$}
        \question{If both roots are real and are of opposite sign, then complete set of values of $a$ is}
        \choice{(0,3)}{($-\infty$, 0) \(\cup\) (3, \(\infty\))}{$\left\{\frac{1}{2}\right\}$}{$\left\{1,3\right\}$}

        \question{If $a$ is given as $\frac{n^2+n}{2}$, where $n$ is natural number then both the roots are necessarily}
        \choice{Integers}
        {Rational number}
        {Even integers}
        {Odd integers}

        \vspace{5mm}
        {As per the data of previous question both the root will lie in interval}
        \choice{$[-1$, 2]}
        {$[1$, $\infty$]}
        {$[1$, 2]}
        {$\left[\frac{1}{2}, 2\right]$}

        \vspace{3mm}
        \hrule
        \vspace{3mm}
        \item[\textbf{Question:}]{$f(x)=x^3+ax^2+bx+c$ has 3 distinct positive integral roots $\alpha$, $\beta$, $\gamma$ such that $\alpha < \beta < \gamma$. Also $g(x)=x^2-2x+79$ such that $f(g(x))=0$ has no real roots. Also $f(78)=77$.}
        \question{$\alpha+\beta+\gamma$ is}
        \choice{A prime number}
        {A multiple of 5}
        {A perfect square}
        {Divisible by exactly 4 natural numbers}

        \question{$c$ is equal to}
        \choice{$-366289$}
        {$-364089$}
        {Odd number}
        {Multiple of 3}

        \question{Choose the incorrect statement}
        \choice{Out of $\alpha$, $\beta$, $\gamma$ exactly 2 are prime}
        {$(\alpha -3)$ is a perfect square}
        {Number of divisors of $\gamma$ is 2}
        {$\beta\gamma$ is a perfect square}
        \vspace{3mm}
        \hrule
        \vspace{3mm}
        \question{Let $p$, $q$, $r$ are the roots of the equation $x^3 +ax^2+bx+c=0$ and $\alpha$, $\beta$, $\gamma$ are the roots of the equation $x^3+2x^2(3p+a)+4x(3p^2+2ap+b)+8(p^3+ap^2+bp+c)=0$}\vspace{2mm}
        {(1) One out of $\alpha$, $\beta$, $\gamma$ is $2q-2p$}\vspace{1mm}\\
        \vspace{1mm}{(2) Sum of the roots $\alpha$, $\beta$ and $\gamma$ is $2(q+r)-p$}\\
        \vspace{1mm}{(3) One of the root out of $\alpha$, $\beta$, $\gamma$ is $2p-q$}\\
        \vspace{1mm}{(4) Product of roots $\alpha$, $\beta$, $\gamma$ is 0}\\

        \question{Let $z$ be a complex number on the locus $\frac{z-\dot{\imath}}{z+\dot{\imath}}=e^{\dot{\imath}\theta}(\theta \in R)$, such that $|z-3-2\dot{\imath}|+|z+1-3\dot{\imath}|$ is minimum, then which of the following statement(s) is(are) correct?}
        \choice{$\arg(z)=0$}{$\arg(z)=\pi$}
        {$|z|=\frac{7}{5}$}{$z=7$}

      \end{questions}
\end{document}


   
